\letter{
\begin{flushleft}
\setlength{\parindent}{9pt}
Dear Members of the Application Review Committee,
I am writing in response to your advertisement for the Staff Physicist position (\#418580).

I am a research associate working at Fermilab on the CMS experiment and 2025 LHC Physics Center (LPC) Distinguished Researcher.

My research tackles fundamental open questions of the Standard Model (SM)—our best framework for describing nature, yet still incomplete—through dedicated and novel searches for its extensions and the development of cutting-edge silicon detectors to enable unprecedented exploration of collider physics. In this position, I aim to ensure we fully exploit the opportunities of the upcoming High Luminosity LHC (HL-LHC) era and contribute to the next generation of collider experiments.


Over the years, I have worked extensively with {\bf silicon detectors,} including planar and 3D pixel sensors and strip sensors. Since my first internship at Fermilab in 2015, I have been involved in the {\bf CMS Phase-2 tracker upgrades}. In my current role, I work on the CMS Outer Tracker upgrade. The future detector will allow particle momentum discrimination at the L1 trigger-for the first time at a hadron collider-thanks to modules composed of two closely spaced silicon sensors and multiple ASICs. I play a key role in module testing, calibration development, and production readiness, including serving on a task force to ensure the functionality of these objects before the start of production. In this position, I would have the extraordinary opportunity to contribute to the {\bf ATLAS tracking detector} during its critical construction phase, helping to ensure it delivers high-quality data throughout the HL-LHC era. I am particularly eager to lead the development and production of the thin-wall titanium cooling tubes, which are essential to meeting the low-mass and high-performance requirements of the ATLAS ITk, while being actively involved in the module production. I intend to take initiative and contribute meaningfully to these efforts, fostering the next generation of detector experts.

I am looking forward to contributing to the {\bf R\&D efforts on Monolithic Active Pixel Sensors (MAPS)}, an emerging silicon technology suited for a diverse set of applications, including at a future muon collider or CERN's Future Circolar Collider (FCC) and even beyond colliders. While I am already studying available prototypes, I plan to leverage the expertise of the {\bf microelectronics group}, and {\bf develop a design suited for future high-energy physics experiments}, playing a key role in {\bf shaping detectors for future colliders}.

The coming years are crucial for completing the upgrades of detectors for the High Luminosity
phase and shaping the design of future collider experiments. With my extensive expertise in
silicon detector technologies, quality assurance tools, and calibration systems, I am well-
positioned to take on leadership roles in these endeavors.

I have led searches for new particles predicted by several SM extensions, focusing on hadronic final states, developing a method to probe multiple SM extensions in one search with unprecedented sensitivity, adaptable to more exotic scenarios. Now, I focus on an Effective Field Theory (EFT) approach to study Vector Boson Scattering (VBS), a key probe of electroweak (EW) interactions through quartic gauge couplings. I am using this channel to search for anomalies in these couplings within the EFT framework. 
A {\bf major goal of the precision LHC physics program} is to to explore the EW and Higgs sectors particularly through {\bf longitudinal W boson scattering cross-section measurements}.  So far, these studies have focused on leptonic decays, but including hadronic decays would greatly enhance sensitivity. However, jet polarization remains largely unexplored. I aim to leverage Argonne’s expertise in SM, EFT, jet physics, and ML to develop a method for W polarization discrimination in hadronic decays, providing new insights into HWW couplings, complementing direct searches, and taking full advantage of the HL-LHC dataset.

Beyond research, I am passionate about {\bf outreach}. I engaged with both students at the lab and the local community. Recently, I took part in the organization of the Conference for Undergraduate Women and Gender Minorities in Physics (CU*iP) at Fermilab. Beyond the lab, I have given a talk at a local high school, invited by a former mentee, and judged the Afro-Academic, Cultural, Technological, and Scientific Olympics (ACT-SO) since 2022. Moreover, I provided {\bf workshops for advanced students}, including the CMS Data Analysis School and the Excellence in Detector and Instrumentation Technologies (EDIT) school. I am committed to {\bf continuing impactful outreach efforts} and to {\bf equipping PhD students with essential research and technical skills, fostering collaborations with academic institutions}.

Argonne’s multidisciplinary environment is particularly exciting, and I am eager to contribute to its innovative efforts in advancing our understanding of the universe.


%In addition, I am passionate about {\bf outreach}, inspiring young people to see themselves as scientists and experience the excitement of discovery, especially those who may not have had these opportunities before. I engaged with both students at the lab and the local community. I have led tours of the Silicon Detector Facility for various lab programs and recently took part in the organization of the Conference for Undergraduate Women and Gender Minorities in Physics (CU*iP) at Fermilab. Beyond the lab, I have given a talk at a local high school, invited by a former mentee, and judged the Afro-Academic, Cultural, Technological, and Scientific Olympics (ACT-SO) since 2022. Moreover, I provided {\bf workshops for advanced students}, including the CMS Data Analysis School (DAS) and the Excellence in Detector and Instrumentation Technologies (EDIT) school. I am committed to {\bf continuing outreach efforts} that bring science to those who may not have been exposed to the excitement of advancing our understanding of the world and that {\bf equip PhD candidates with essential skills fostering collaborations with academic institutions}.

%Argonne's dynamic and multidisciplinary research environment is particularly exciting, offering the opportunity to apply my expertise to cutting-edge projects. I am eager to contribute to its innovative efforts in advancing our understanding of the universe. Additionally, I find it enormously inspiring to be in a rich, diverse, and multicultural setting. The Argonne environment and its multiple research fields, to which I would contribute through my own experiences and academic studies, are exceptionally attractive and stimulating.

Thank you for considering my application. I am available for any additional information or clarification you may need. I look forward to the opportunity to discuss this further.

Respectfully yours,

Irene Zoi
\end{flushleft}

}