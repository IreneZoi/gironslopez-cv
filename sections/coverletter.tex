\letter{
\begin{flushleft}
\setlength{\parindent}{10pt}
Dear Members of the Application Review Committee,

%I am writing in response to the advertisement for the Wilson Fellowship position (R\_008330).
I am a Fermilab research associate working on the CMS experiment and 2025 LHC Physics Center (LPC) Distinguished Researcher.

My research is dedicated to answering the Standard Model (SM) open questions through dedicated and novel searches for its extensions and the development of cutting-edge silicon detectors. The upcoming years are fundamental to make sure collider physics experiments keep pushing the boundaries of our knowledge. As a Wilson Fellow, I will ensure we fully exploit the HL-LHC era and contribute to designing the next generation of collider experiments.

%I have worked extensively with various silicon detector technologies, including planar and 3D pixel sensors and strip sensors. Currently, I play a central role to the Phase-2 CMS Outer Tracker upgrade, including serving on a task force to ensure the functionality of these objects before the start of production and focusing on module testing, calibration, and production readiness. I have been involved in the CMS Phase-2 tracker upgrades since my first time at Fermilab as an intern in 2015. Ten years later, as a Wilson Fellow, I will have the extraordinary opportunity and challenge to finally build the detector and ensure it will deliver high quality data during the HL-LHC era.


%I have led searches for new resonances in hadronic final states, developing jet-substructure techniques to probe multiple BSM scenarios. Now, I am focusing on EFT-based searches in the Vector Boson Scattering (VBS) channel, incorporating polarization studies to explore both the electroweak and Higgs sectors. This is one of the main objectives of the precision LHC physics program and has never been exploited in final states where vector bosons decay to jets, that thanks to their high branching fraction increase the chances of measuring this rare process. Combining the Fermilab’s expertise in SM and EFT physics, jet substructure and ML with my understanding of the process, as a Wilson Fellow I can develop a technique to distinguish longitudinally and transversely polarized W bosons decaying hadronically in a boosted jet, allowing insights into HWW couplings, complementing direct searches. 
Several theoretical models addressing the SM open questions predict the existence of new particles. I have led searches able to probe several models at once with unprecedented sensitivity. Now, I focus on an Effective Field Theory (EFT) approach which provides a model-independent way to parametrize potential deviations from the SM. Vector Boson Scattering (VBS) is one of the few processes probing gauge boson self-interactions through quartic gauge couplings and thus is key to testing the electroweak (EW) sector. I am now using this channel to perform a search for anomalies in the couplings, modeled through the EFT framework. 
One of the {\bf main objectives of the precision LHC physics program} is to to explore the EW and Higgs sectors. Both can be assessed measuring the scattering {\bf cross-section of longitudinally polarized W bosons.} This is so far being done only in the case in which the bosons decay to leptons. VBS is a very rare process and including the case where one of the bosons decays to quarks will greatly enhance the reach of the studies. However, jet polarization is largely unexplored. As a Wilson Fellow, I aim to leverage Fermilab’s expertise in SM and EFT physics, jet physics, and ML to {\bf develop a method for distinguishing W polarization in hadronic decays}, providing new {\bf insights into HWW couplings, complementing direct searches, and benefiting from the high branching ratio of this decay channel}. %Moreover, the {\bf observation of the VBS production of WV bosons in the semi-leptonic final states} can be achieved combining Run-2 and Run-3 datasets.

I have worked extensively with {\bf silicon detectors,} including planar and 3D pixel sensors and strip sensors. Since my first internship at Fermilab in 2015, I have been involved in the {\bf CMS Phase-2 tracker upgrades}. In my current role, I work on the CMS Outer Tracker upgrade. The future detector will allow particle momentum discrimination at the L1 trigger thanks to modules composed of two closely spaced silicon sensors and multiple ASICs. I play a key role in module testing, calibration development, and production readiness, including serving on a task force to ensure the functionality of these objects before the start of production. As a Wilson Fellow, I have the extraordinary opportunity to {\bf build the detector}, ensure it delivers high-quality data for the HL-LHC era, and foster the next generation of experts.


The coming years are critical for designing the next generation of detectors for future colliders. As a Wilson Fellow, I will lead Fermilab’s {\bf R\&D efforts on Monolithic Active Pixel Sensors (MAPS)}, an emerging silicon technology suited for a diverse set of applications, including at a future muon collider or CERN's Future Circolar Collider (FCC). While I am already studying available prototypes, I plan to contribute to securing a contract with a {\bf U.S.-based foundry to strengthen domestic semiconductor and MAPS development}. Leveraging the expertise of the {\bf microelectronics group}, I aim to {\bf develop a design suited for future high-energy physics experiments}, ensuring Fermilab’s Collider Physics Division plays a key role in {\bf shaping detectors for future colliders}.


%My research focuses on silicon detectors and extensions of the standard model. 
%The next years are crucial to make sure we can continue addressing the remaining open questions in an effective way. As a Wilson Fellow I will ensure that we exploit the full potential of the upcoming HL-LHC era and design the next generation of collider experiments.
%
%
%During my studies, I gained experience with different silicon detector technologies, including planar and 3D pixel sensors and strip sensors. In my current role, I work on the Phase-2 upgrade of the CMS Outer Tracker. The future detector will allow particle momentum discrimination at the L1 trigger thanks to modules composed of two closely spaced silicon sensors (strip-strip or pixel-strip) and multiple ASICs. I have performed extensive module testing and calibration development, including serving on a task force to ensure the functionality of these objects before the start of production. Additionally, I support an R\&D effort at Fermilab focused on Monolithic Active Pixel Sensors (MAPS), an exciting new area of silicon detector technology.
%The coming years are crucial for completing the upgrades of detectors for the High Luminosity phase and shaping the design of future collider experiments. With my extensive expertise in silicon detector technologies, quality assurance tools, and calibration systems, I am well- positioned to take on leadership roles in these endeavors.
%My analysis experience covers searches for new resonances in hadronic final states. Among the final states covered by the search, there is the decay to a pair of Vector Bosons and a Higgs and a Vector Boson. As the leading analyzer, I tested different jet-substructure techniques and developed a method that allows, with one single search, to probe many proposed extensions and can be further adapted to test even more exotic scenarios. Now, I am also performing searches in an EFT context. My current interest is in developing measurements in the VBS channel, combined with polarization studies of the vector bosons, as a way to probe not only the EW sector but also the Higgs one. This is one of the main objectives of the precision LHC physics program and has never been fully exploited in final states where vector bosons decay to jets. Given my previous experience, I could lead this effort and provide insight into the HWW couplings in a complementary approach to direct searches.

%I am passionate about outreach, especially in inspiring young people who may not realize they can be scientists or experience the excitement of discovery. Fermilab has given me the opportunity to engage with both students at the lab and the local community. I have led tours of the Silicon Detector (SiDet) facility for various lab programs and recently served on the Program Committee for CU*iP held at Fermilab in January 2025.
%Beyond the lab, I have given a talk at a local high school, invited by a former mentee, and have been a judge in the Engineering category for ACT-SO\footnote{Afro-Academic, Cultural, Technological, and Scientific Olympics} since 2022. I am committed to continuing outreach efforts that bring science to those who may not have been exposed to the excitement of advancing our understanding of the world. I will also keep providing workshops and tutorials for advanced students, such as the CMS Data Analysis School or the EDIT school, equipping PhD candidates with the tools and knowledge necessary to succeed in their doctoral studies and fostering collaborations with academic institutions.

\vskip 10pt
I am passionate about {\bf outreach}, inspiring young people to see themselves as scientists and experience the excitement of discovery, especially those who may not have had these opportunities before. {\bf Fermilab has given me the opportunity to engage with both students at the lab and the local community}. I have led tours of the Silicon Detector Facility for various lab programs and recently took part in the organization of the Conference for Undergraduate Women and Gender Minorities in Physics (CU*iP) 2025 at Fermilab. I am also on the Fermilab Society of Women Engineers (fSWE) Executive Board, assigned to the Outreach Subcommittee. Beyond the lab, I have given a talk at a local high school, invited by a former mentee, and judged the Afro-Academic, Cultural, Technological, and Scientific Olympics (ACT-SO)’s Engineering category since 2022. Moreover, I  provided {\bf workshops for advanced students}, including CMS Data Analysis School (DAS) and the Excellence in Detector and Instrumentation Technologies (EDIT) school. I am committed to {\bf continuing outreach efforts} that bring science to those who may not have been exposed to the excitement of advancing our understanding of the world and that {\bf equip PhD candidates with essential skills and foster collaborations with academic institutions}.

Thank you for considering my application. I am at your disposal for any further information or explanation.

Respectfully yours,

Irene Zoi
\end{flushleft}

}