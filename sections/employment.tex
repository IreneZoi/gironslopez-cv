% Professional experience
\ifswedish
  \section{Arbetslivserfarenhet}
    \position
      {aug 2018 \textemdash{} }
      {Hydrolog}
      {Sveriges meteorologiska och hydrologiska institut, SMHI (Sverige)}
      {Tjänsteman för den svenska och europeiska översvämningsvarningstjänsten och forskare om hydrologiska säsonsprognoser.}
    \position
      {feb 2017 \textemdash{} aug 2018}
      {Postdoctorsforskare}
      {Zürich universitet (Schweiz)}
      {Min forskning fokuserar på att förbättra realismen av HBV-modellen snörutin, utan att äventyra den betecknande låg modell osäkerhet. Också, jag bidrog till glacio-hydrologisk modellering av framtida höga flöden i avlägsna dalar i Tadzjikistan.}
    \position
      {aug 2011 \textemdash{} okt 2016}
      {Lärarassistent}
      {Uppsala universitet (Sverige)}
      {Föreläsning, ledning av seminarier, och ledning och assistans i laboratorie- och fältarbete både i grund- och forskarsutbildning.}
    \position
      {aug 2011 \textemdash{} okt 2016}
      {Forskningsassistent}
      {Uppsala universitet (Sverige)}
      {Utformning och genomförande av forskningsprojekt inom hydrologi, inklusive definition av forskningsfrågor, modellering, dataanalys och rapportering av vetenskapliga resultat.}
    \position
      {jun 2009 \textemdash{} aug 2009}
      {Praktikant inom kvalitetskontroll}
      {Laboratori Català de Control (Spanien)}
      {Kvalitetskontroll av cement, betong och förankringar. Provning av markegenskaper, vattenprovtagning och analys. Begränsningen av förfaranden för att anta ISO-regler.}
    \position
      {jul 2008 \textemdash{} sep 2008}
      {Praktikant inom miljökonsult}
      {URS Corporation (Spanien)}
      {Miljökontrolltest av bensinstationer, markdekontaminering och hydrogeologisk databehandling. Fauna- och sedimentprovtagning för vattenkvalitetskontroll av floder och dammar.}
\else
  \section{Professional Experience}
   
  \position
  {08/2025  \textemdash{} now} 
   {Application Physicist 1}
   {Fermilab}
{\begin{itemize}

\item Leading the testing of {\bf strip-strip and pixel-strip modules} for the {\bf Outer Tracker Phase-2 upgrade} during production. The modules are made of {\bf two closely spaced silicon sensors} and {\bf multiple ASICs} to provide {\bf tracking information to the L1 trigger} and {\bf on module p$_{T}$ discrimination} for the {\bf first time at a hadron collider}.
\begin{itemize}
\item Supervision of students and postdocs during module, electronics and sensor testing
\item Development of the DAQ tools for testing and qualification of the modules
\end{itemize}
\item Coordinating a team of various institutes working on Vector Boson Scattering in final states with at least on boson decaying to jets
\begin{itemize}
\item Supervision of a grad student to develop {\bf polarization studies in jets} %the same final state
\end{itemize}
\item Supporting role in the {\bf R\&D of Monolithic Active Pixels Sensor} with prospects for {\bf future colliders}
\item Coordinating searches for heavy resonances, expanding my previous efforts
\item {\bf Supervision and mentoring of students}, presenting results at international conferences
\item Active member of the {\bf CMS} collaboration at {\bf CERN}  
   
 \end{itemize}
}
  
   
  \position
  {11/2021  \textemdash{} 08/2025} 
   {Research associate}
   {Fermilab}
{\begin{itemize}

\item Leading contribution in testing and calibration development of {\bf strip-strip and pixel-strip modules} for the {\bf Outer Tracker Phase-2 upgrade}.
\begin{itemize}
\item Responsible for DAQ and p$_{T}$ discrimination tools for an irradiated pixel-strip module testbeam at Fermilab to verify the p$_{T}$ discrimination capability
\item Leading contribution in pixel-strip module task force solving outstanding issues before the start of the module production
\item Developer modules calibrations and Q\&A tools, for the production of thousands of modules
\end{itemize}
\item Leading the {\bf search for anomalous EW production of VBS WV (semi-leptonic) with dimension-8 EFT operators}
\begin{itemize}
\item Supervision of a grad student to develop {\bf polarization studies in jets} %the same final state
\end{itemize}
\item Supporting role in the {\bf R\&D of Monolithic Active Pixels Sensor} with prospects for {\bf future colliders}
\item Coordinating searches for heavy resonances, expanding my previous efforts
\item {\bf Supervision and mentoring of students}, presenting results at international conferences,  involved in {\bf LPC events and committees}
\item Active member of the {\bf CMS} collaboration at {\bf CERN}

\end{itemize}
}

    \position
      {02/2017 \textemdash{} 10/2021}
      {Research assistant}
      {Universit\"{a}t Hamburg (Germany)}
      {\begin{itemize}
\item Active member of the {\bf CMS} collaboration at {\bf CERN}
\item {\bf Contact person and main data analyst} of a search for new particles covering a large variety of di-boson (W, Z, H) all-hadronic final states and {\bf  jet-tagging techniques}
\item Took part in several {\bf test beam} campaigns at the {\bf DESY} and {\bf Fermilab} facilities, {\bf collecting and analysing the data}, to characterize and select {\bf pixel silicon sensors} for the {\bf Phase-2 upgrade}
%\item Monitoring of the {\bf radiation damage} in the current pixel detector, pixel simulation contact and pixel on-call shifter during data taking periods
\item Supervision of students, presenting results at international conferences
\end{itemize}
}
    \position
      {11/2016 \textemdash{} 01/2017}
      {Research assistant}
      {Universit\`a degli studi di Firenze (Italy)}
      {{\bf R\&D of planar and 3D pixel sensors} for the CMS Phase-2  Upgrade}
%      {\begin{itemize}
%\item {\bf R\&D of planar and 3D pixel sensors} for the CMS Phase-2  Upgrade
%\item Collaborating with the group for several months after I moved to Hamburg to pass the knowledge acquired at Fermilab to their new students and to finalize the results I presented at EPS-HEP 2017
%\end{itemize} 
%}

    \position
      {04/2016 \textemdash{} 10/2016}
      {Visiting Scientist}
      {Fermilab (USA)}
      {Characterization of non-irradiated and irradiated pixel sensor properties through beam and laser tests, calibrations and analysis of the collected data.}
    \position
      {07/2015 \textemdash{} 09/2015}
      {Summer Intern}
      {Fermilab (USA)}
      {Analysis of test beam data of pixel sensor prototypes.}
\fi
