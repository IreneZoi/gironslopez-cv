% Professional experience
\ifswedish
  \section{Arbetslivserfarenhet}
    \position
      {aug 2018 \textemdash{} }
      {Hydrolog}
      {Sveriges meteorologiska och hydrologiska institut, SMHI (Sverige)}
      {Tjänsteman för den svenska och europeiska översvämningsvarningstjänsten och forskare om hydrologiska säsonsprognoser.}
    \position
      {feb 2017 \textemdash{} aug 2018}
      {Postdoctorsforskare}
      {Zürich universitet (Schweiz)}
      {Min forskning fokuserar på att förbättra realismen av HBV-modellen snörutin, utan att äventyra den betecknande låg modell osäkerhet. Också, jag bidrog till glacio-hydrologisk modellering av framtida höga flöden i avlägsna dalar i Tadzjikistan.}
    \position
      {aug 2011 \textemdash{} okt 2016}
      {Lärarassistent}
      {Uppsala universitet (Sverige)}
      {Föreläsning, ledning av seminarier, och ledning och assistans i laboratorie- och fältarbete både i grund- och forskarsutbildning.}
    \position
      {aug 2011 \textemdash{} okt 2016}
      {Forskningsassistent}
      {Uppsala universitet (Sverige)}
      {Utformning och genomförande av forskningsprojekt inom hydrologi, inklusive definition av forskningsfrågor, modellering, dataanalys och rapportering av vetenskapliga resultat.}
    \position
      {jun 2009 \textemdash{} aug 2009}
      {Praktikant inom kvalitetskontroll}
      {Laboratori Català de Control (Spanien)}
      {Kvalitetskontroll av cement, betong och förankringar. Provning av markegenskaper, vattenprovtagning och analys. Begränsningen av förfaranden för att anta ISO-regler.}
    \position
      {jul 2008 \textemdash{} sep 2008}
      {Praktikant inom miljökonsult}
      {URS Corporation (Spanien)}
      {Miljökontrolltest av bensinstationer, markdekontaminering och hydrogeologisk databehandling. Fauna- och sedimentprovtagning för vattenkvalitetskontroll av floder och dammar.}
\else
  \section{Professional Experience}
    \position
      {02/2017 \textemdash{} now}
      {Research assistant}
      {Universit\"{a}t Hamburg (Germany)}
      {\begin{itemize}
\item Active member of the {\bf CMS} collaboration at {\bf CERN}
\item {\bf Contact person} of a search for new particles covering a large variety of di-boson (W, Z, H) all-hadronic final states and {\bf  jet/b-tagging techniques}
\item Successfully studied the {\bf performance} of {\bf jet substructure observables} to identify {\bf highly boosted W/Z bosons}
\item Took part in several {\bf test beam} campaigns at the {\bf DESY} and {\bf Fermilab} facilities to characterize {\bf pixel silicon sensors} for the {\bf Phase-2 upgrade}
\item Monitoring of the radiation damage in the current pixel detector, pixel simulation contact and pixel on-call shifter during data taking periods
\item Supervision of students, presenting results at international conferences
\end{itemize}
}
    \position
      {04/2016 \textemdash{} 10/2016}
      {Visiting Scientist}
      {Fermilab (USA)}
      {Characterization of non-irradiated and irradiated pixel sensor properties trough beam and laser tests, calibrations and analysis of the collected data.}
    \position
      {07/2015 \textemdash{} 09/2015}
      {Summer Intern}
      {Fermilab (USA)}
      {Analysis of test beam data of pixel sensor prototypes.}
\fi
