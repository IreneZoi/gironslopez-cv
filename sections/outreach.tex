\section{Outreach Statement}
\vskip 20pt
\descript{ \large
\begin{flushleft}
Looking back at how I chose to start a scientific career, I recognize that some of the decisive moments happened well before I entered university and that I was fortunate during my school education. I recall at least two important episodes. In the last years of high school, I could spend two weeks at the Arcetri Astrophysical Observatory in Florence, where I attended some lectures, did some small research exercise, and talked to the scientists working there. During my physics classes in high school, I didn't just solve the textbook exercises, but I received an introduction to modern physics, with practical examples of how famous scientists planned and realized experiments and compared the results to the theory.  We also learned how ideas evolved in time, for instance, how the atom concept changed from the simple plum pudding model to the non-intuitive Schrödinger probabilistic orbitals. 

\vspace{\baselineskip}
The opportunity to see science in action with my own eyes was a real game-changer for which I am truly grateful.
Thus, I would like new generations to experience the same inspiration.
Throughout my Ph.D., I took part in two outreach programs dedicated to high school students. In both cases, I supervised a small group of students carrying out a research project. Thanks to this experience, I improved my communication skills by adapting my explanations to a non-expert audience, transforming complicated ideas into simple and fascinating concepts. I prepared exercises to introduce the students to coding, planned activities to keep them involved, and to show them different aspects of scientific research, from using an experimental setup to data analysis. 

\vspace{\baselineskip}

The commitment of the Fermilab Lederman Fellowship to outreach activities is remarkably appealing, as it would support my effort in engaging new students in science and research. 
Having participated during my Ph.D. in educational programs and conferences, also addressing undergraduates, I developed teaching skills and learned how to adjust my exposition to the audience level and interests. Now, I would be glad to put the acquired know-how into new outreach campaigns, such as the Saturday Morning Physics program. 
The diverse set of activities carried out at Fermilab, and the laboratory involvement in the territory life and different aspects of society make it a uniquely attractive place to bring my contribution to making physics more accessible for everybody. 
\vspace{\baselineskip}


Social media are also a powerful tool to educate and inspire people of all ages, especially in the current times, where everything has moved online.
I have already been active on online platforms by writing two posts describing my work for pages featuring scientists of different fields, \href{https://www.instagram.com/p/CEcnln7DIN-/}{``Women Doing Science"} and \href{https://www.instagram.com/p/B-HEgJEDnBm/}{``{\em Le storie della ricerca}"} (Research stories). Receiving positive feedback, I am strongly motivated to invest more time in promoting science and publishing educational material through these channels.
\end{flushleft}
}
