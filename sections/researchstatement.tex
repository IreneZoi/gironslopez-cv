\section{Research Statement}
\vskip 20pt
\descript{ \large
\begin{flushleft}


Particle physics experiments are delivering significant results testing the predictions of the Standard Model (SM) and exploring its extensions. These studies have opened the path to even more exciting research in the upcoming years. The Higgs boson has been discovered, but its properties, still under investigation, could show discrepancies from the SM expectations. Neutrino masses are yet to be measured, their hierarchy has to be established, and observed anomalies need to be clarified. Hints to new physics, addressing the remaining open questions, can be found by performing precise SM measurements and exploring the broad set of Beyond the Standard Model (BSM) scenarios, both in the context of specific models and effective field theory (EFT) frameworks. % FIXME -> focus on EFT!! and jets!! and VBS

\vspace{\baselineskip}

Vector boson scattering (VBS) and vector boson fusion (VBF) processes can be studied at collider experiments as CMS. They are crucial to investigate the electroweak (EW) sector and are sensitive to new physics at energy scales beyond the LHC direct reach through an EFT description of the SM.

VBS measurements probe the non-Abelian gauge structure of the electroweak interaction, which leads to self-interactions between gauge bosons via triple and quartic gauge couplings. Potential variations of these couplings, due to physics beyond the standard model, could increase the cross section of the VBS production of vector bosons at high energies, producing changes in the tail of high-energy distributions, with respect to the SM expectation.

I am leading a search for anomalous electroweak production of WV bosons in association with two jets, in the case where the W boson decays leptonically and the V (either a W or a Z) decays hadronically.
The latest CMS result\footnote{CMS Collaboration, {\em Search for anomalous electroweak production of vector boson pairs in association with two jets in proton-proton collisions at 13 TeV}, Physics Letters B, Volume 798, 2019, Article 134985, \href{https://doi.org/10.1016/j.physletb.2019.134985}{doi:10.1016/j.physletb.2019.134985}} was obtained with 35.9~fb$^{-1}$ of data collected in 2016. The search will now benefit from the full 138~fb$^{-1}$ of data collected in the 2016--2018 data taking period and improved calculations of the dimension-8 EFT operators.

Measurements in the VBS channel combined with polarization studies of the vector bosons are another way to probe not only the EW sector but also the Higgs one.
The measurement of the  W$^L_+$W$^L_+$ scattering cross section highly depends on the HWW couplings\footnote{Michal Szleper, {\em The Higgs boson and the physics of WW scattering before and after Higgs discovery}, \href{https://arxiv.org/pdf/1412.8367}{arXiv:1412.8367}} providing a complementary approach to the determination of the couplings. 

I am part of an LPC based team that together with a graduate student from the University of Virginia is aiming at studying the polarization of two W bosons produced in association with two jets. The effort targets the semileptonic final state to take advantage of the high branching ratio of the W $\to q\bar{q}$ and to explore polarization in jets, developing dedicated machine learning (ML) techniques. 

\vspace{\baselineskip}
 
The LPC has shown an increased interest in EFT physics and the determination of organizing yearly workshops on the topic. The VBS and polarization research presented above would expand the offered program and would benefit from the connections and expertise the LPC provides.

The next workshop is already planned to take place again at Fermilab. With my previous experience in the LPC event committee and my involvement in EFT activities I would contribute to a successful realization. 

A critical challenge for many analyses probing higher dimension operators is the sample generation taking into account the different reweighing weights. With the knowledge gained from my current effort I could assist new analysts in this delicate and crucial step to develop a new study.

The study of polarization in jets would benefit from the LPC interest in ML and its applications for jet tagging or anomaly detection bringing a stimulating application.

\vspace{\baselineskip}

%FIXME: here I need something to transition to the detector part
\vspace{\baselineskip}
% FIXME -> update to OT, production etc.
In the next months, the CMS collaboration will enter a crucial stage of the detectors' developments for the HL - LHC. I am working on the Outer Tracker (OT) for the Phase-2 upgrade.
The design of the new tracker has been driven by the necessity to provide, for the first time in CMS, tracking information to the L1 trigger, allowing trigger rates to be kept at a sustainable level without sacrificing physics potential. For this, the OT will be made out of p$_{\mathrm{T}}$-modules with two closely spaced silicon sensors read out by front-end ASICs, which can correlate hits in the two sensors creating short track segments called ``stubs''. The modules come in two flavors: strip-strip (2S) and pixel-strip (PS), which contain different sensor configurations and multiple ASICs. 

Fermilab is an assembly and testing center for both module types. I am involved in the testing of the modules and in developing the qualification and grading procedures. Giving the delicate task the modules perform for the stub identification and the presence of multiple chips on every module, several calibration steps are foreseen to evaluate the noise performance and a proper communication among all components. 

About 5000 p$_{\mathrm{T}}$-modules (considering 2S and PS types) will be produced in the US across several institutes, and half of them at Fermilab. The pre-production is about to start and the production will be at full rate during the next DR mandate. This is a crucial period to train student and postdocs in testing the modules. It is expected that many of them will come to the LPC to help with the testing. With my experience I can prepare training material, instruct the future module testers from US institutes, and coordinate their efforts.

Creating expertise is crucial also to guarantee a smooth transition to the following integration, installation and operation phases. 



 \vspace{\baselineskip}
 
Both my analysis and technical work are crucially related to the activities on-going at Fermilab. As a DR I would benefit even more by the vibrating environment of the LPC  complementing the physics research already performed and supporting its training mission with a special dedication to the instrumentation sector.

I am involved in mentoring students of all levels, from high school to PhD, coming from different backgrounds and I am providing, for instance, tours of the SiDet,  the Silicon detector facility, for many programs at the lab. Through my experience and passion I hope to get minorities more involved in physics research and to provide them with useful insights and a welcoming environment to successfully pursue their interests.


\end{flushleft}
}
