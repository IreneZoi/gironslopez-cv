\section{Research Statement}
\vskip 10pt
\descript{ \large
\begin{flushleft}
\setlength{\parindent}{20pt}

Particle physics experiments are delivering significant results testing the predictions of the Standard Model (SM) and exploring its extensions. These studies have opened the path to even more exciting research in the upcoming years. The Higgs boson has been discovered, but its properties, still under investigation, could show discrepancies from the SM expectations. Hints to new physics, addressing the remaining open questions, can be found by performing precise SM measurements and exploring the broad set of Beyond the Standard Model (BSM) scenarios, both in the context of specific models and effective field theory (EFT) frameworks. In parallel, the use of cutting-edge technology in the detectors and the need to always improve their performance has driven innovation with applications extending far beyond high-energy physics.

My research is dedicated to the development of advanced detectors that enable unprecedented exploration of collider physics while fostering new applications across diverse domains. I am passionate about unraveling the complexities of the SM and its extensions through meticulous and innovative studies.
\vspace{\baselineskip}
\section{Silicon detectors development}
\vspace{\baselineskip}
During my academic journey, I focused on silicon detector technologies, including planar and 3D pixel sensors, strip sensors, and monolithic sensors. My work included evaluating the radiation hardness and spatial resolution of small-pitch pixel sensors, a cornerstone of the CMS Phase-2 Inner Tracker upgrade. Currently, I am deeply involved with the CMS Outer Tracker  (OT) Phase-2 upgrade. I have started to engage with the R\&D of Monolithic Active Pixel Sensors (MAPS), which could play a key role in future collider experiments.
\vspace{\baselineskip}
\subsection{The Outer Tracker Phase-2 upgrade}
\vspace{\baselineskip}
In the coming months, the CMS collaboration will enter a critical phase of detector developments for the HL-LHC. I am actively involved in the Outer Tracker (OT) upgrade for Phase-2. This new tracker is designed to provide, for the first time in CMS, tracking information to the Level-1 (L1) trigger, enabling sustainable trigger rates without sacrificing physics potential. The OT will be composed of p$_{\mathrm{T}}$-modules, which will provide a rough transverse momentum measurement to the L1 trigger. These modules consist of two closely spaced silicon sensors read out by front-end ASICs, which correlate hits in the sensors to create short track segments known as ``stubs.'' The modules come in two types: strip-strip (2S) and pixel-strip (PS), each with distinct sensor configurations and multiple ASICs.

In 2023 I led a successful testbeam campaign at Fermilab to verify the PS module stub identification procedure and to compare the performance of non-irradiated and irradiated sensors. The preparation for the testbeam included the development of several tools needed to debug the stub performance and to calibrate the module.

The Silicon Detector facility (SiDet) at Fermilab serves as an assembly and testing center for both module types. I have also conducted extensive lab testing on both module types and developed quality control procedures to ensure proper communication between components and optimizing noise performance. These tests are critical for establishing grading procedures that will allow a reliable operation of the upgraded OT under the high-radiation conditions of the LHC. 

Given the critical role of these modules in stub identification and the complexity introduced by multiple chips per module, a complete set of quality control tests is implemented for the modules and their components to evaluate noise performance and ensure proper communication among all components. Therefore, all modules are required to undergo calibration and tests while performing multiple thermal cycles from room temperature to the operation temperature (around -35$^{\circ}$C) over 24 hours to check for possible early mortality. The collected data will then be used to identify the best modules to be installed in the detector. 

As member of a dedicated task force to solve outstanding issues with PS modules, I have achieved stable operation at the working temperature of about -35$^{\circ}$C and correct stubs readout for all module flavors, including the first observation of stubs with the fastest readout PS modules.  

\vspace{\baselineskip} 
\subsection{Monolithic Active Pixel Sensors R\&D}
\vspace{\baselineskip}
Monolithic Active Pixel Sensors (MAPS) are a highly promising technology with vertex detection, tracking, and calorimetry applications. They offer extremely high granularity over large areas through stitching, combining low power density with cost-effectiveness. MAPS are notable for their minimal mass, crucial for precision in high-energy physics experiments.
Current developments focus on enhancing spatial resolution, timing, rate capabilities, and radiation hardness. These improvements ensure MAPS can handle fast, high-rate events and withstand harsh environments. Additionally, MAPS development is closely linked with advancements in ASICs, optimizing overall detector performance through co-design procedures.

I recently became involved in the R\&D for these versatile detectors that could have a variety of applications in future collider experiments. The studies focus on the ARCADIA prototype, featuring a patterned backside to achieve uniform full depletion over thicknesses of few hundreds microns. The activities involve support for a recent testbeam to assess the performance of three MAPS prototypes for the ALICE tracker upgrade that took place in the Fermilab Test Beam Facility. Moreover, a test setup is present at SiDet that features a laser for charge collection studies. This will allow to compare the simulation of the device physics with bench and testbeam measurements.


%\vspace{\baselineskip}
\vspace{\baselineskip}
\section{Investigating the Standard Model open questions}
\vspace{\baselineskip}
To address the open questions of the Standard Model, I have been both pursuing searches for particles predicted by promising BSM theories and tackling an EFT approach.

Several theoretical models address the shortcomings of the SM predicting the existence of new particles with masses of the order of the TeV that can be produced at colliders. I developed a method that allows, with one single search, to probe many proposed extensions and can be further adapted to test even more exotic scenarios\footnote{CMS Collaboration, {\em Search for new heavy resonances decaying to WW, WZ, ZZ, WH, or ZH boson pairs in the all-jets final state in proton-proton collisions at $\sqrt{s}=13$ TeV}, Phys. Lett. B 844 (2023) 137813, \href{http://dx.doi.org/10.1016/j.physletb.2023.137813}{doi.:10.1016/j.physletb.2023.137813}}. Moreover, the method’s flexibility permitted to perform the first combined search for heavy resonances decaying in hadronic WW, WZ, ZZ, WH and ZH final states. The bosons produced in the decay of these massive resonances have a high momentum. Thus, they decay into two collimated quarks, and the sprays of particles produced in the quarks' hadronization are merged into a single large radius jet. Therefore, jet substructure techniques are employed to identify jets originating from vector bosons or containing b-quark pairs from jets stemming from a single quark or gluon. Algorithms based on machine learning (ML) techniques are used to achieve the highest discriminating power. This study’s background estimation and signal extraction procedures rely on a multidimensional maximum likelihood fit to the masses of the jets originating from the bosons and their combined dijet system. The signal is resonant in the considered dimensions and a three-dimensional search for a bump on smoothly falling backgrounds is performed. A decorrelation method has been validated to prevent ML-based algorithms from introducing sculpting in the shape of the jet mass used as observable in the fit.
\vspace{\baselineskip}
\subsection{VBS measurements to probe EFT and expand the Higgs sector}
\vspace{\baselineskip}
Vector boson scattering (VBS) processes at collider experiments are crucial for investigating the electroweak (EW) sector. VBS is sensitive to new physics at energy scales beyond the direct reach of the Large Hadron Collider (LHC), allowing researchers to probe potential deviations from the SM. VBS processes involve the self-interactions of gauge bosons through triple and quartic gauge couplings, central features of the SM's non-Abelian gauge structure. Changes in these couplings could lead to observable effects, such as increased cross-sections in high-energy VBS production, offering clues about BSM physics.

A recent CMS publication has measured evidence for the production of WV bosons in association with two jets, in the case where the W boson decays leptonically and the V boson (either a W or a Z) decays hadronically with 4.4$\sigma$ significance\footnote{CMS Collaboration, {\em Evidence for WW/WZ vector boson scattering in the decay channel $\ell\nu$qq produced in association with two jets in proton-proton collisions at $\sqrt{s}$ = 13 TeV}, Phys. Lett. B 834 (2022) 137438, \href{https://doi.org/10.1016/j.physletb.2022.137438}{doi.:10.1016/j.physletb.2022.137438}}. I am now leading a search for the anomalous electroweak production of such process. The latest CMS result\footnote{CMS Collaboration, {\em Search for anomalous electroweak production of vector boson pairs in association with two jets in proton-proton collisions at 13 TeV}, Phys. Lett. B, 798 (2019), 134985, \href{https://doi.org/10.1016/j.physletb.2019.134985}{doi:10.1016/j.physletb.2019.134985}} was obtained with 35.9~fb$^{-1}$ of data collected in 2016. This search will now benefit from the full 138~fb$^{-1}$ of data collected in the 2016--2018 data taking period, as well as improved calculations of the dimension-8 EFT operators. It is planned to publish a combined search of the WZ and ZV (where the Z decays leptonically) anomalous VBS production, in which the sensitivity to the EFT operators is driven by the WV channel. 

Measurements in the VBS channel, combined with polarization studies of the vector bosons, are another way to probe not only the EW sector but also the Higgs one.
The measurement of the  W$^L_+$W$^L_+$ scattering cross-section is highly dependent on the HWW couplings\footnote{Michal Szleper, {\em The Higgs boson and the physics of WW scattering before and after Higgs discovery}, \href{https://arxiv.org/pdf/1412.8367}{arXiv:1412.8367}} providing a complementary approach to determining these couplings. 
I am part of a team aiming to study the polarization of two W bosons produced in association with two jets. This effort targets the semi-leptonic final state to take advantage of the high branching ratio of W $\to q\bar{q}$ decays and to explore polarization in jets, developing dedicated machine learning (ML) techniques. 


With the availability of the full Run 2 and Run 3 statistics, the first observation for the VBS production of WV bosons in the semi-leptonic final states is within reach and with my experience in this channel I am well positioned to achieve it.
While polarization studies have primarily focused on fully leptonic final states, exploring semi-leptonic final states could offer higher statistics due to the large branching ratio of W $\to q\bar{q}$ decays. Drawing on my expertise in jet substructure, I aim to develop innovative tools based on state-of-the-art jet tagging techniques to enable detailed polarization studies in jets.
 


\vspace{\baselineskip}
\section{Summary} 
\vspace{\baselineskip}
The coming years are pivotal for completing the upgrades of LHC detectors for the High Luminosity phase and shaping the designs of future collider experiments. With my extensive expertise in silicon strip and pixel detector technologies, quality assurance tools, and calibration systems, I am well-positioned to take on leadership roles in these endeavors. My experience equips me to contribute meaningfully to the production, qualification and integration phases of the upcoming upgrades of the detectors for the HL-LHC phase.
Looking ahead, my dedication to R\&D in MAPS positions me to drive innovations in this cutting-edge technology. These advancements will not only benefit collider experiments but also open new frontiers in broader applications, such as medical imaging, space exploration, and industrial technologies. By leveraging my skills and vision, I aspire to bridge the gap between high-energy physics research and its transformative applications, fostering a deeper understanding of the universe.
In addition, I look forward to applying my experience to tackle unanswered questions of the Standard Model through searches for new particles and EFT approaches, while leveraging my knowledge of jet substructure techniques to drive innovative projects that push the boundaries of discovery.

\end{flushleft}
}
