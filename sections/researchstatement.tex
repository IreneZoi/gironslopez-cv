\section{Research Statement}
\vskip 10pt
\descript{ \large
\begin{flushleft}
\setlength{\parindent}{20pt}

Particle physics experiments are delivering significant results testing the predictions of the Standard Model (SM) and exploring its extensions. These studies have opened the path to even more exciting research in the upcoming years. The Higgs boson has been discovered, but its properties, still under investigation, could show discrepancies from the SM expectations. Hints to new physics, addressing the remaining open questions, can be found by performing precise SM measurements and exploring the broad set of Beyond the Standard Model (BSM) scenarios, both in the context of specific models and effective field theory (EFT) frameworks. % FIXME -> focus on EFT!! and jets!! and VBS

Vector boson scattering (VBS) and vector boson fusion (VBF) processes can be studied at collider experiments as CMS. They are crucial to investigate the electroweak (EW) sector and are sensitive to new physics at energy scales beyond the Large Hadron Collider (LHC) direct reach through an EFT description of the SM. 
In the last years the LHC Physics Center has shown a significant interest to EFT. As a 2025 Junior Distinguished Researcher (DR) I would expand the program with a focus on VBS measurements and a complementary approach to the study of the Higgs properties.

My technical work is centered around the module production and qualification for the Outer Tracker Phase-2 upgrade. Fermilab is one of the US assembly and testing centers and will be in the module production phase next year. As a DR I will lead the training and coordination of the numerous students and postdocs that are expected to come to the LPC to test the modules that will be produced.

%\vspace{\baselineskip}

\section{VBS measurements to probe EFT and expand the Higgs sector}

\vspace{\baselineskip}
VBS measurements probe the non-Abelian gauge structure of the electroweak interaction, which results in self-interactions between gauge bosons via triple and quartic gauge couplings. Potential variations in these couplings, due to physics beyond the standard model, could increase the cross-section of the VBS production of vector bosons at high energies, leading to observable changes in the high-energy distribution tails compared to SM expectations.
A recent CMS publication has measured evidence for the production of WV bosons in association with two jets, in the case where the W boson decays leptonically and the V boson (either a W or a Z) decays hadronically with 4.4$\sigma$ significance\footnote{CMS Collaboration, {\em Evidence for WW/WZ vector boson scattering in the decay channel $\ell\nu$qq produced in association with two jets in proton-proton collisions at $\sqrt{s}$ = 13 TeV}, Phys. Lett. B 834 (2022) 137438, \href{https://doi.org/10.1016/j.physletb.2022.137438}{doi.:10.1016/j.physletb.2022.137438}}. I am now leading a search for the anomalous electroweak production of such process. The latest CMS result\footnote{CMS Collaboration, {\em Search for anomalous electroweak production of vector boson pairs in association with two jets in proton-proton collisions at 13 TeV}, Phys. Lett. B, 798 (2019), 134985, \href{https://doi.org/10.1016/j.physletb.2019.134985}{doi:10.1016/j.physletb.2019.134985}} was obtained with 35.9~fb$^{-1}$ of data collected in 2016. This search will now benefit from the full 138~fb$^{-1}$ of data collected in the 2016--2018 data taking period, as well as improved calculations of the dimension-8 EFT operators. It is planned to publish a combined search of the WZ and ZV (where the Z decays leptonically) anomalous VBS production, in which the sensitivity to the EFT operators is driven by the WV channel. The analysis is in good shape to reach approval to be presented at conferences in the first half of 2025.

Measurements in the VBS channel, combined with polarization studies of the vector bosons, are another way to probe not only the EW sector but also the Higgs one.
The measurement of the  W$^L_+$W$^L_+$ scattering cross-section is highly dependent on the HWW couplings\footnote{Michal Szleper, {\em The Higgs boson and the physics of WW scattering before and after Higgs discovery}, \href{https://arxiv.org/pdf/1412.8367}{arXiv:1412.8367}} providing a complementary approach to determining these couplings. 
I am part of an LPC-based team, collaborating with a graduate student from the University of Virginia, aiming to study the polarization of two W bosons produced in association with two jets. This effort targets the semileptonic final state to take advantage of the high branching ratio of W $\to q\bar{q}$ decays and to explore polarization in jets, developing dedicated machine learning (ML) techniques. The analysis will feature both Run 2 and Run 3 data-taking periods. As the publication plan for this analysis is still being developed, inputs from the LPC Higgs and EFT groups, as well as the broad Fermilab theory community, will lead to a sound result that benefits all involved parties.
 
The LPC has demonstrated a growing interest in Effective Field Theory EFT physics, and it is committed to organizing yearly workshops on the topic. The VBS and polarization research I am leading would significantly enhance the program, leveraging the extensive connections and expertise that the LPC offers.
The next workshop is scheduled to take place at Fermilab. With my previous experience on the LPC event committee and my active involvement in EFT activities, I am well-positioned to contribute to its successful execution.
A critical challenge for many analyses probing higher-dimension operators is the generation of samples that accurately account for different reweighing schemes. Drawing from the insights gained through my current efforts, I can provide valuable assistance to new analysts in navigating this complex and crucial process. Moreover, collaborating with other analysis groups at the LPC, we can develop common EFT tools and reinforce the LPC's role as a center for precision EW physics.
The study of polarization in jets stands to benefit from the LPC's expertise in machine learning (ML) and expand its applications in jet tagging and anomaly detection. This intersection presents a stimulating opportunity to apply cutting-edge ML techniques to enhance our understanding of jet polarization.


%\vspace{\baselineskip}
\section{Module production and qualification for the Outer Tracker Phase-2 upgrade}
\vspace{\baselineskip}
In the coming months, the CMS collaboration will enter a critical phase of detector developments for the HL-LHC.  I am actively involved in the Outer Tracker (OT) upgrade for Phase-2. This new tracker is designed to provide, for the first time in CMS, tracking information to the Level-1 (L1) trigger, enabling sustainable trigger rates without sacrificing physics potential. The OT will be composed of p$_{\mathrm{T}}$-modules featuring two closely spaced silicon sensors read out by front-end ASICs, which correlate hits in the sensors to create short track segments known as ``stubs.'' The modules come in two types: strip-strip (2S) and pixel-strip (PS), each with distinct sensor configurations and multiple ASICs.

In 2023 I led a successful testbeam campaign at Fermilab to verify the PS module stub identification procedure and to compare the performance of non-irradiated and irradiated sensors. Several students and postdocs from both US and international institutions contributed to data collection and analysis. Many of them continue to visit or stay at the LPC to prepare the paper summarizing our promising results. As a DR, I can ensure proper supervision of the students and timely completion of the publication.

The Silicon Detector facility (SiDet) at Fermilab serves as an assembly and testing center for both module types. My work focuses on testing these modules and developing qualification and grading procedures. As member of a dedicated task force to solve outstanding issues with PS modules, I have achieved stable operation at the working temperature of about -35$^{\circ}$C and correct stubs readout for all module flavors, including the first observation of stubs with the fastest readout PS modules.

Given the critical role of these modules in stub identification and the complexity introduced by multiple chips per module, a complete set of quality control tests is implemented for the modules and their components to evaluate noise performance and ensure proper communication among all components. A precise alignment of the sensors is also crucial for the correct stub reconstruction. Therefore, all modules are required to undergo calibration and tests while performing multiple thermal cycles from room temperature to the operation temperature (around -35$^{\circ}$C) over 24 hours to check for possible early mortality. The collected data will then be used to identify the best modules to be installed in the detector.
 
Approximately 5000 p$_{\mathrm{T}}$-modules (including both 2S and PS types) will be produced in the US, with half of them assembled at Fermilab. As pre-production is about to commence and full-scale production is expected during the next DR mandate, this period is crucial for training students and postdocs in module testing. Many of these trainees are anticipated to come to the LPC for hands-on experience and to support the extensive qualification process.
Leveraging my expertise, I can prepare comprehensive training materials, instruct future module testers from US institutes, and coordinate their efforts. Developing this competence is essential for ensuring a successful completion of the upgraded tracker and a smooth transition to the subsequent integration, installation, and operation phases of the detector upgrade.

%\vspace{\baselineskip}
\section{Monolithic Active Pixel Sensors R\&D}
\vspace{\baselineskip}
Monolithic Active Pixel Sensors (MAPS) are a highly promising technology with vertex detection, tracking, and calorimetry applications. They offer extremely high granularity over large areas through stitching, combining low power density with cost-effectiveness. MAPS are notable for their minimal mass, crucial for precision in high-energy physics experiments.
Current developments focus on enhancing spatial resolution, timing, rate capabilities, and radiation hardness. These improvements ensure MAPS can handle fast, high-rate events and withstand harsh environments. Additionally, MAPS development is closely linked with advancements in ASICs, optimizing overall detector performance through co-design procedures.

I recently became involved in the R\&D for these versatile detectors that could have a variety of applications in future collider experiments. A collaboration of US and international institutions is involved in the program. While preparing this application, a successful testbeam to assess the performance of three MAPS prototypes for the ALICE tracker upgrade took place in the Fermilab Test Beam Facility. Moreover, a test setup is present at SiDet and a laser for charge  collection studies is being added. This will allow to compare the simulation of the device physics with bench and testbeam measurements.
As on-site activities ramp up, being a DR in 2025 would allow me to take advantage from the support of the LPC community to enhance Fermilab's contribution to this effort by bringing together and growing the newly formed R\&D collaboration.

 
\section{LPC activities}
\vspace{\baselineskip}
I have been a member of the LPC community since beginning my postdoc in November 2021. Prior to that, I benefited from the training opportunities and stimulating environment at LPC as a student.
My contributions to the LPC include serving several committees and facilitating the yearly Data Analysis School (DAS). 
As a member of the LPC event committee, I helped organize the first LPC EFT workshop at Fermilab and the Open Data workshop. 
Shortly after joining Fermilab at the end of 2021, I proposed and successfully implemented ideas for social events at the online DAS in 2022.
In the 2023 DAS, I led the jet tagging short exercise and facilitated the H$\to$WW long exercise. In 2024, I facilitated two short exercises and, on very short notice, adapted a planned long exercise to be accessible to the talented students who placed third in the final competition.

I currently serve as the chair of the Physics Forum committee that provides a variety of topics to a broad audience, often beyond the CMS experiment community. As a 2025 DR, I plan to maintain my responsibilities as Physics Forum chair and play a significant role in organizing LPC activities, including the DAS, the Hands-on Advanced Tutorial Session (HATS) and the next EFT workshop. I am actively involved in mentoring students at all levels, from high school to PhD, and from diverse backgrounds. Moreover, I provide tours of the SiDet facility for various laboratory programs. Through my experience and passion, I aim to increase the involvement of minorities in physics research, offering them valuable insights and a welcoming environment to successfully pursue their interests.



\section{Summary} 
\vspace{\baselineskip}
Both my analysis and technical work are crucially related to the activities on-going at Fermilab. As a DR I would benefit even more from the vibrating environment of the LPC  complementing the EFT physics research already performed with a search for anomalous quartic gauge couplings and polarization studies in semileptonic VBS signatures. Moreover, I would be supporting its training mission with a special dedication to the instrumentation sector, becoming more and more crucial as we approach the production stage of the Phase-2 upgrade and we develop new detectors for future colliders. I would supervise the students and postdoc expected to come to the LPC for the testing of the hundreds of Outer Tracker modules to be produced and expand the Monolithic Active Pixel Sensors R\&D starting at SiDet.



\end{flushleft}
}
