\section{Statement of Research Interests}
\vskip 5pt
\descript{ \small
Particle physics experiments are delivering significant results testing the predictions of the Standard Model (SM) and exploring its extensions, opening the path to even more exciting research in the upcoming years. The Higgs boson has been discovered, but its properties, still under investigation, could show discrepancies from the SM expectations. Hints to new physics, addressing the remaining open questions, can be found by performing precise SM measurements and exploring the broad set of Beyond the Standard Model (BSM) theories providing a diverse and attractive collection of phenomena that experiments could observe.
Working at CERN, I would take an active part in the challenge of extending our understanding of the constituents of our universe. This Fellowship will allow me to take great advantage of the laboratory expertise, covering all aspects of the experimental effort: from data analysis to detector development.
The LHC data offer the unique opportunity to develop innovative analysis' methods to perform the high precision measurements required to discover new physics and forces us to build new sophisticated detectors.
To increase our particle physics knowledge, I am motivated to dedicate effort to different aspects of the experimental work, maintaining a broad picture while addressing a specific topic.

While pursuing my PhD working on the Compact Muon Solenoid (CMS) experiment, I participated both in searches for new resonances and in the development of new detectors. During my previous studies at the University of Florence and research stays at Fermilab, I grew a strong background in silicon sensors and a desire to understand for what kind of physics research they could be applied. 

Various SM extensions, attempting to explain the discrepancies between the electroweak and the gravitational scales and the need for fine-tuned SM parameter to obtain the measured Higgs mass, predict the existence of heavy resonances. Many such models favor the decay of the new particles to two bosons (W, Z, or Higgs), which decay with a high branching fraction to pairs of quarks, merged into one "boosted" jet due to the large resonance mass. Thus, during my PhD, I carried out searches for these resonances addressing a broad set of hadronic final states and studied jet substructure variables' performance to overcome the difficulty of identifying highly boosted W, Z, and Higgs bosons. I first completed a full new analysis focused on resonances produced through Vector Boson Fusion (VBF), characterized by two additional jets. The cross-section of this process is non-negligible, but the impact of polarization effects between the resonance spin and the extra jets, changing the event's shape, makes the analysis extremely challenging. Then, I joined the effort to extend a search targeting resonances decaying to a pair of Vector Bosons to include also final states with a Higgs and a Vector Boson, becoming analysis contact and representing the analysis through CMS internal review. I applied a novel approach that extracts the signal using a three-dimensional maximum likelihood fit of the two jet masses and the dijet invariant mass. This method successfully improved the sensitivity of the search in final states with two Vector Bosons\footnote{
%CMS Collaboration, {\em A multi-dimensional search for new heavy resonances decaying to boosted WW, WZ, or ZZ boson pairs in the dijet final state at 13TeV}, 
Eur. Phys. J. C (2020) 80:237}.
I am consolidating the technique and proving its efficacy in a broader variety of all-jets final states with bosons, including WH and ZH.
I tested several jet-substructure approaches to understand which one would give the  most reliable performance. Deep-Neural-Network (DNN) taggers provide the best signal to background discrimination but would introduce features in the jet mass distribution, precluding this variable's use in the fit. To employ these taggers and achieve the best sensitivity in the analysis, I am commissioning a DNN taggers' decorrelation technique. Furthermore, I adapted the background estimation method to include the extended signal region's additional contributions by developing their 3-dimensional modeling. 
With my experience in tagging boosted jets from Higgs and Vector Bososns, as well as VBF signatures, I want to contribute to decisive measurements performed in the Higgs and SM sectors, such as di-Higgs and boosted Higgs measurements or Vector Boson Scattering (VBS) and multibosons topologies. 


Besides, I further fostered my knowledge of pixel detectors, which already included a comprehensive study of silicon sensor properties, with the characterization of small pitch prototypes' spatial resolution for the CMS Phase-2 Upgrade at the High Lumi (HL) - LHC, addressing the effect of radiation damage. To cope with the high pile-up and radiation conditions expected at the HL-LHC, the CMS Pixel detector's granularity will be six times higher than the current one and sensors must be radiation hard. Employing a dedicated approach, I measured the spatial resolution of pixel prototypes, contributing to the selection of the final sensor design for the future detector.


%Precise measurement of , crucial for the reconstruction of variables and particles' momentum for many SM and BSM measurements, profits with reduced pixel size but degrades with irradiation: By employing   
The LHC experiments are in a crucial phase of the developments for the HL - LHC.  It would be extremely stimulating and exciting to bring my input to these relevant projects and contribute to the new detectors' construction. 

 Being involved in different projects, I consolidated my previous knowledge and acquired additional expertise. Now I am eager to find new applications for what I learned and get involved in state-of-the-art projects that will bring progress to the field while expanding my knowledge: The diverse cutting-edge research possibilities carried out at CERN constitute the perfect environment to achieve it. 


%With the start of Run 3 postponed to 2022, 2021 gives the opportunity to improve analysis techniques and adapt them to a broader set of possibilities while getting ready for the new data taking period. 



%Furthermore, the search was still carried out under the assumption that the new resonances' width is smaller than the detector energy resolution. I would be interested in further developing the approach to consider also larger signal widths, predicted by an increasing number of theoretical models, as well as more generic hadronic final states.


%In the next years, the CMS collaboration will enter a crucial phase of the detectors' developments for the HL - LHC.  It would be extremely stimulating and exciting to bring my input to these relevant projects and contribute to the new detectors' construction. Having followed the pixel Phase-2 upgrade closely by test beam campaigns and dedicated studies, I realized how crucial it is to thoroughly test all the new components. Furthermore, it is important to integrate into the simulation studies for the HL - LHC the promising sensors and elements currently under test to verify that the detector design fulfills the requirements and delivers the expected physics results. Also, understanding and simulating the radiation damage at the demanding HL-LHC fluences represents a challenging and necessary topic to address. 


%Finally, working in collaboration with the CERN group will allow me to closely follow the new detector's commissioning, operation, and monitoring to gain the relevant expertise to ensure excellent data quality and physics performance in the promising future upgrade and to play a significant role in the progression of our knowledge with groundbreaking research.
}