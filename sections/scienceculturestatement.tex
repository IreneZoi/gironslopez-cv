\section{Science Culture Statement}
\vskip 10pt
\descript{ \large
\begin{flushleft}
\setlength{\parindent}{20pt}
%Science Culture statement not to exceed two (2) pages. Examples of topics might include discussion of the diversity, equity, inclusion within our scientific community and/or broader impacts such as outreach, education, environment and advocacy. 

%\vspace{\baselineskip}
%\section{Silicon detectors development}

\vspace{\baselineskip}
\section{Promoting Diversity, Equity, and Inclusion (DEI) in Science}
\vspace{\baselineskip}
I am deeply committed to fostering a culture of diversity, equity, and inclusion within the scientific community and the local area. I believe that diverse teams whose members have unique perspectives and experiences lead to innovative ideas. To this end, I actively support initiatives aimed at broadening participation in science, particularly among historically underrepresented groups. %This includes mentoring students from diverse backgrounds, engaging in recruitment efforts that prioritize equity, and advocating for inclusive policies at all levels of the scientific enterprise.
I have participated in several initiatives targeting gender minorities students. These include proposing a 3 day exercise on Silicon Photon  Multiplier for high school students while I was a PhD student in Germany or providing tours of the Silicon Detector (SiDet) facility at Fermilab for the SAGE program. Moreover, I am part of the Program Committee of the Conference for Undergraduate Women and Gender Minorities in Physics held at Fermilab in January 2025. My efforts are not limited to the communities in which I identify with as a women in science but also to other minorities. Since 2022 I have been a judge for the Afro-Academic, Cultural, Technological, and Scientific Olympics (ACT-SO) in the Engineering category. The program targets African American High school students in the DuPage county giving them research experience and awards to help them enter competitive college programs.

In my own work, I strive to create an environment where everyone feels valued and empowered to contribute. I aim to ensure that all voices are heard and respected. I recognize that achieving true equity requires not only removing barriers but also actively dismantling systemic inequities that persist within the scientific community. I am part of several mentoring programs to provide guidance and advice to the new generation of scientists.

\vspace{\baselineskip}
\section{Broader Impacts Through Outreach and Education}
\vspace{\baselineskip}
Science outreach and education are integral to building a society that values and understands scientific inquiry. Throughout my career, I have participated in numerous outreach initiatives designed to inspire the next generation of scientists. 
I was very pleased to be invited by one student that worked with my group for several months to give a talk at her highschool portraying an example of a woman in a STEM field. It was a great opportunity to explain to young people about to decide their future directions how a day as a scientist can look like, what did I do to get there and the other possibilities they may have. I am very passionate about providing tours of SiDet to students, showing them state-of-the-art devices on which they may end up working on! 

I have also led workshops and tutorials, as the CMS Data Analysis School, targeting more advance students, providing them with the tools and resources they will need to achieve their doctoral degrees. 
%These include organizing public talks, developing educational materials for schools, and leading hands-on workshops that make complex scientific concepts accessible to diverse audiences.

%One of my most fulfilling outreach efforts has been engaging with students in underserved communities. By introducing them to the excitement of research through interactive demonstrations and discussions, I hope to spark their curiosity and encourage them to consider careers in STEM fields. I also actively collaborate with educators to integrate real-world science into classroom curricula, emphasizing the relevance of scientific discovery to everyday life.

\vspace{\baselineskip}
\section{Environmental Responsibility in Scientific Practice}
\vspace{\baselineskip}
As scientists, we have a responsibility to minimize the environmental impact of our work. I am committed to adopting sustainable practices in my research and promoting environmental stewardship within the broader scientific community. This includes optimizing laboratory processes to reduce waste, advocating for the use of renewable energy sources in research facilities, and supporting policies that address the environmental challenges associated with large-scale scientific experiments. While pushing for change on a large scale, I recognize that there are practices we can follow on an everyday basis. I am aware of the impacts that actions like sending computing jobs have and I am committed to an efficient use of the available resource.

%Furthermore, I believe that science has a vital role to play in addressing global environmental issues. By conducting research that informs sustainable development and climate policy, I aim to contribute to a more sustainable future. I also prioritize interdisciplinary collaborations that leverage expertise across fields to tackle complex environmental challenges.

\vspace{\baselineskip}
\section{Advocacy for Science and Society}
\vspace{\baselineskip}

During my time spent at the laboratory I have learned the value of advocating for the importance of science in society and with policymakers. 
As an international researcher, I felt I needed to get familiar with the system I am working in to achieve a positive outcome when engaging with funding agencies. Through this position I  will now have the possibility to advocate for sustained investment in research and education.
%Engaging with funding agency is not something I was asked to do. is a cornerstone of my professional mission. I regularly engage with policymakers, funding agencies, and the public to highlight the value of scientific research and its transformative potential.

I am particularly passionate about ensuring that scientific knowledge is accessible to all. This involves not only disseminating research findings through open-access publications but also translating complex concepts into clear, relatable language. By bridging the gap between scientists and the public, I hope to build trust and foster a shared commitment to addressing the pressing challenges of our time.

\vspace{\baselineskip}
\section{Conclusion}
\vspace{\baselineskip}
In summary, I believe I have a responsibility in ensuring an inclusive and diverse community of scientist can provide the new ideas and contributions to science. I recognize the value of providing education to students, especially from underrepresented communities, and of advocating for research and sustainability. With my concrete efforts and actions, I aim to contribute to a scientific community that is not only innovative but also equitable and impactful.

\end{flushleft}
}
