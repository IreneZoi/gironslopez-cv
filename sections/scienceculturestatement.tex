\section{Science Culture Statement}
\vskip 10pt
\descript{ \large
\begin{flushleft}
\setlength{\parindent}{20pt}
%Science Culture statement not to exceed two (2) pages. Examples of topics might include discussion of the diversity, equity, inclusion within our scientific community and/or broader impacts such as outreach, education, environment and advocacy. 

%\vspace{\baselineskip}
%\section{Silicon detectors development}

\vspace{\baselineskip}
\section{Promoting Diversity, Equity, and Inclusion (DEI) in Science}

As a scientist, I am deeply committed to fostering a culture of diversity, equity, and inclusion within our community. I believe that the most impactful and innovative science arises from diverse teams that bring unique perspectives and experiences to the table. To this end, I actively support initiatives aimed at broadening participation in science, particularly among historically underrepresented groups. This includes mentoring students from diverse backgrounds, engaging in recruitment efforts that prioritize equity, and advocating for inclusive policies at all levels of the scientific enterprise.

In my own work, I strive to create an environment where everyone feels valued and empowered to contribute. Whether through collaborative projects, workshops, or conferences, I aim to ensure that all voices are heard and respected. I recognize that achieving true equity requires not only removing barriers but also actively dismantling systemic inequities that persist within the scientific community.

\vspace{\baselineskip}
\section{Broader Impacts Through Outreach and Education}

Science outreach and education are integral to building a society that values and understands scientific inquiry. Throughout my career, I have participated in numerous outreach initiatives designed to inspire the next generation of scientists. These include organizing public talks, developing educational materials for schools, and leading hands-on workshops that make complex scientific concepts accessible to diverse audiences.

One of my most fulfilling outreach efforts has been engaging with students in underserved communities. By introducing them to the excitement of research through interactive demonstrations and discussions, I hope to spark their curiosity and encourage them to consider careers in STEM fields. I also actively collaborate with educators to integrate real-world science into classroom curricula, emphasizing the relevance of scientific discovery to everyday life.

\vspace{\baselineskip}
\section{Environmental Responsibility in Scientific Practice}

As scientists, we have a responsibility to minimize the environmental impact of our work. I am committed to adopting sustainable practices in my research and promoting environmental stewardship within the broader scientific community. This includes optimizing laboratory processes to reduce waste, advocating for the use of renewable energy sources in research facilities, and supporting policies that address the environmental challenges associated with large-scale scientific experiments.

Furthermore, I believe that science has a vital role to play in addressing global environmental issues. By conducting research that informs sustainable development and climate policy, I aim to contribute to a more sustainable future. I also prioritize interdisciplinary collaborations that leverage expertise across fields to tackle complex environmental challenges.

\vspace{\baselineskip}
\section{Advocacy for Science and Society}

Advocating for the importance of science in society is a cornerstone of my professional mission. I regularly engage with policymakers, funding agencies, and the public to highlight the value of scientific research and its transformative potential. This includes communicating the societal benefits of scientific advancements and advocating for sustained investment in research and education.

I am particularly passionate about ensuring that scientific knowledge is accessible to all. This involves not only disseminating research findings through open-access publications but also translating complex concepts into clear, relatable language. By bridging the gap between scientists and the public, I hope to build trust and foster a shared commitment to addressing the pressing challenges of our time.

\vspace{\baselineskip}
\section{Conclusion}

In summary, my vision for a vibrant science culture centers on inclusivity, education, sustainability, and advocacy. By promoting diversity, engaging with the public, prioritizing environmental responsibility, and championing the role of science in society, I aim to contribute to a scientific community that is not only innovative but also equitable and impactful.

\end{flushleft}
}
