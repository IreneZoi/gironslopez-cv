\section{Science Culture Statement}
\vskip 10pt
\descript{ \large
\begin{flushleft}
\setlength{\parindent}{20pt}
%Science Culture statement not to exceed two (2) pages. Examples of topics might include discussion of the diversity, equity, inclusion within our scientific community and/or broader impacts such as outreach, education, environment and advocacy. 

%\vspace{\baselineskip}
%\section{Silicon detectors development}

\vspace{\baselineskip}
\section{Promoting Diversity, Equity, and Inclusion (DEI) in Science}
\vspace{\baselineskip}
I am deeply committed to fostering a culture of diversity, equity, and inclusion within the scientific community and beyond. I believe that diverse teams, enriched by unique perspectives and experiences, lead to innovative ideas. To this end, I actively support initiatives aimed at broadening participation in science, particularly among historically underrepresented groups. %This includes mentoring students from diverse backgrounds, engaging in recruitment efforts that prioritize equity, and advocating for inclusive policies at all levels of the scientific enterprise.
My efforts have included proposing a three-day exercise on Silicon Photon Multipliers for high school students during my PhD in Germany and providing tours of the Silicon Detector (SiDet) facility at Fermilab for the SAGE program. Recently, I served on the Program Committee for the Conference for Undergraduate Women and Gender Minorities in Physics, held at Fermilab in January 2025. 
Beyond my identity as a woman in science, I strive to support other underrepresented communities. Since 2022 I have been a judge for the Afro-Academic, Cultural, Technological, and Scientific Olympics (ACT-SO) in the Engineering category. This program empowers African American high school students in DuPage County with research opportunities and awards to help them gain access to competitive college programs.

In my own work, I push to create an environment where all voices are valued and respected.  I recognize that achieving true equity involves not just removing barriers but actively dismantling systemic inequities. I participate in mentoring programs to guide and inspire the next generation of scientists.

\vspace{\baselineskip}
\section{Broader Impacts Through Outreach and Education}
\vspace{\baselineskip}
Science outreach and education are integral to building a society that values and understands scientific inquiry. Throughout my career, I have actively participated in initiatives designed to inspire the next generation of scientists.
 
One highlight was being invited by a former mentee to give a talk at her high school, where I shared my experiences as a woman in STEM. This was a wonderful opportunity to demystify the path to becoming a scientist and to inspire young people to pursue careers in science. Additionally, I am very passionate about providing SiDet tours to students, showing them state-of-the-art devices on which they may end up working on! 

I have also led workshops and tutorials, such as the CMS Data Analysis School, where I guided advanced students, equipping them with the tools and knowledge necessary to succeed in their doctoral studies.
%These include organizing public talks, developing educational materials for schools, and leading hands-on workshops that make complex scientific concepts accessible to diverse audiences.

%One of my most fulfilling outreach efforts has been engaging with students in underserved communities. By introducing them to the excitement of research through interactive demonstrations and discussions, I hope to spark their curiosity and encourage them to consider careers in STEM fields. I also actively collaborate with educators to integrate real-world science into classroom curricula, emphasizing the relevance of scientific discovery to everyday life.

\vspace{\baselineskip}
\section{Environmental Responsibility in Scientific Practice}
\vspace{\baselineskip}
As scientists, we bear a responsibility to minimize the environmental impact of our work. I am committed to adopting sustainable practices in research and promoting environmental stewardship within the scientific community. This includes optimizing laboratory processes to reduce waste, advocating for renewable energy in research facilities, and supporting policies to mitigate the environmental footprint of large-scale experiments.
On a practical level, I am mindful of the environmental impacts of routine actions, such as computational resource usage, and strive for efficiency in these areas.

%Furthermore, I believe that science has a vital role to play in addressing global environmental issues. By conducting research that informs sustainable development and climate policy, I aim to contribute to a more sustainable future. I also prioritize interdisciplinary collaborations that leverage expertise across fields to tackle complex environmental challenges.

\vspace{\baselineskip}
\section{Advocacy for Science and Society}
\vspace{\baselineskip}

My time at the laboratory has shown me the value of advocating for science within society and engaging with policymakers. As an international researcher, I have worked to familiarize myself with the systems that govern scientific funding and policy. Through this position I  will now have the possibility to advocate for sustained investment in research and education.
%Engaging with funding agency is not something I was asked to do. is a cornerstone of my professional mission. I regularly engage with policymakers, funding agencies, and the public to highlight the value of scientific research and its transformative potential.

I am particularly passionate about ensuring that scientific knowledge is accessible to all. This involves not only disseminating research findings through open-access publications but also translating complex concepts into clear, relatable language. By bridging the gap between scientists and the public, I hope to build trust and foster a shared commitment to addressing the pressing challenges of our time.

\vspace{\baselineskip}
\section{Conclusion}
\vspace{\baselineskip}
In summary, I believe I have a responsibility in ensuring an inclusive and diverse community of scientist can provide the new ideas and contributions to science. I recognize the value of providing education to students, especially from underrepresented communities, and of advocating for research and sustainability. With my concrete efforts and actions,  I aim to leave a positive and lasting impact on science and society.

\end{flushleft}
}
